\chapter{Darstellung der DICOM-Elemente im Speicher} \label{appendix:speicher}
\section{Explizite VR mit [ OB \textpipe\ OW \textpipe\ OF \textpipe\ SQ \textpipe\ UT \textpipe\ UN ]}

Bei expliziter VR-Struktur besteht das Element aus vier konsekutiven Feldern. Ist die VR vom Typ OB, OW, OF, SQ, UT oder UN wird das Datenelement wie in Tabelle \ref{table:appendix_explizit} im Speicher abgelegt. Die reservierten 2 Byte im VR-Teil sind für zukünftige Weiterentwicklungen des DICOM-Standards.\cite[7.1.2]{dicom:structure}

\begin{sidewaystable}
    \begin{tabularx}{\textwidth}{|X|X|p{5cm}|X|X|p{8cm}|}
    \toprule \hline
   \multicolumn{2}{|l|}{\textbf{Tag}} 	&	\multicolumn{2}{l|}{\textbf{VR}} 		&		\textbf{Value Length}   	& 	\textbf{Value} \\ \hline
    Group \# 16-bit unsigned integer & Element \# 16-bit unsigned integer  &  VR 2-byte character String [OB | OW | OF | SQ | UT | UN ] & Reservierter
    Bereich & 32-bit unsigned integer  &  Gerade Anzahl an Byte. Enthält den Wert des Datenelements. Kodierung abhängig von VT-Typ und Transfersyntax. Wenn die Länge nicht definiert ist wird diese auf \glqq Sequence Delimitation\grqq\ limitiert. \\ \hline
	
	2 Byte & 2 Byte & 2 Byte & 2 Byte & 4 Byte & Anzahl an Byte entsprechend der \glqq Value Length\grqq\, wenn von explizieter Länge \\ \hline
	
	\bottomrule
    \end{tabularx}
    \caption {Darstellung des Datenelements im Speicher wenn VR vom Typ OB, OW, OF, SQ, UT oder UN}
    \label{table:appendix_explizit}
\end{sidewaystable}

\section{Explizite VR}

Diese Darstellung wird gewählt wenn VR \textit{nicht} vom Typ OB, OW, OF, SQ, UT oder UN ist. Der Unterschied besteht im Feld \glqq Value Length\grqq\. Bei der Form von Tabelle \ref{table:appendix_explizit} ist dieses Feld 32 Bit lange. Hier beträgt es lediglich 16 Bit \cite[7.1.2]{dicom:structure}. Der Grund liegt am erhöhten Speicherbedarf von \ref{table:appendix_explizit}, da die Länge des Wertes eine undefinierte Länge haben kann.

\begin{sidewaystable}
	\begin{tabularx}{\textwidth}{|X|X|X|X|p{12cm}|}
	\toprule \hline
	\multicolumn{2}{|l|}{\textbf{Tag}} & \textbf{VR} 2 & \textbf{Value Length} & \textbf{Value} 4 \\ \hline
	Group \# 16-bit unsigned integer & Element \# 16-bit unsigned integer & VR 2-byte character String 2 & 16-bit unsigned integer & Gerade Anzahl an Byte. Enthält den Wert des Datenelements. Kodierung abhängig von VT-Typ und Transfersyntax. \\ \hline
	2 Byte & 2 Byte & 2 Byte & 2 Byte & \glqq Value Length\grqq\ Byte \\ \hline
	\bottomrule
	\end{tabularx}
    \caption {Darstellung des Datenelements für alle anderen VR-Typen}
    \label{table:appendix_explizit_else}
\end{sidewaystable}

\section{Implizite VR}

Bei einer impliziten VR Darstellung besteht das Datenelement aus den drei konsekutive Feldern Tag, Value Length und dem Wert selbst \cite[7.1.3]{dicom:structure}.

\begin{sidewaystable}
	\begin{tabularx}{\textwidth}{|X|X|X|p{12cm}|}
	\toprule \hline
	\multicolumn{2}{|l|}{\textbf{Tag}} & \textbf{Value Length} 2 & \textbf{Value} \\ \hline
	Group \# 16-bit unsigned integer & Element \# 16-bit unsigned integer & 32-bit unsigned integer & Gerade Anzahl an Byte. Enthält den Wert des Datenelements. Kodierung abhängig von VT-Typ spezifiziert in \cite{dicom:dd} und Transfersyntax. Wenn die Länge nicht definiert ist wird diese auf \glqq Sequence Delimitation\grqq\ limitiert. \\ \hline
	2 Byte & 2 Byte & 2 Byte & \glqq Value Length\grqq\ Byte oder undefinierte Länge \\ \hline
	
	\bottomrule
	
	\end{tabularx}
    \caption {Darstellung des Datenelements für implizite VR.}
    \label{table:appendix_implizit}
\end{sidewaystable}


